\documentclass[11pt, oneside]{article} % letter, article, report
\usepackage{geometry}                		% See geometry.pdf to learn the layout options. There are lots.
\usepackage{listings}
\usepackage{color}
\geometry{letterpaper}                   		% ... or a4paper or a5paper or ... 
\usepackage{graphicx}				% Use pdf, png, jpg, or eps§ with pdflatex; use eps in DVI mode
								% TeX will automatically convert eps --> pdf in pdflatex		
\usepackage{amssymb}

\definecolor{dkgreen}{rgb}{0,0.6,0}
\definecolor{gray}{rgb}{0.5,0.5,0.5}
\definecolor{mauve}{rgb}{0.58,0,0.82}
\definecolor{black}{rgb}{0,0,0}


\lstset{frame=tb,
  language=Java,
  aboveskip=3mm,
  belowskip=3mm,
  showstringspaces=false,
  columns=flexible,
  basicstyle={\small\ttfamily},
  numbers=none,
  numberstyle=\tiny\color{black},
  keywordstyle=\color{black},
  commentstyle=\color{black},
  stringstyle=\color{black},
  breaklines=true,
  breakatwhitespace=true,
  tabsize=3
}

\title{Nwen 241 C Lab Report}
\author{Diego Trazzi}
%\date{}							% Activate to display a given date or no date




\begin{document}
\maketitle
In this assignment I have learned to pass by reference across methods and the wonderful use of pipes. I really enjoyed learning about pipes, and, if my understanding is correct, pipes are similar to sockets, but for networks, so I hope in future years to learn more about pipes and sockets to make programs more scalable and modular.
\section{Code implementation}
All three parts of the assignment are complete and follow the prototypes provided. I truly hope the code is meeting all the requirements, because I have now re-written it twice: a first version was implemented with the prototypes and logic as I though would make sense for such program, and a second version was then re-written to meet the handout requirements after talking to Ian.
\section{Valgrind}
To test for any memory leek I have used Valgrind to check is the memory allocation was freed before terminating the program. Here is an output showing the commands and the terminal output: 
\begin{lstlisting}
GENERIC
valgrind --tool=memcheck ./tictactoe

==27648== Memcheck, a memory error detector
==27648== Copyright (C) 2002-2013, and GNU GPL'd, by Julian Seward et al.
==27648== Using Valgrind-3.10.0 and LibVEX; rerun with -h for copyright info
==27648== Command: ./tictactoe
==27648== 
This is the game of Tic Tac Toe.
You will be playing against the computer.
How big is your board?
...
==27648== 
==27648== HEAP SUMMARY:
==27648==     in use at exit: 0 bytes in 0 blocks
==27648==   total heap usage: 5 allocs, 5 frees, 96 bytes allocated
==27648== 
==27648== All heap blocks were freed -- no leaks are possible
==27648== 
==27648== For counts of detected and suppressed errors, rerun with: -v
==27648== ERROR SUMMARY: 0 errors from 0 contexts (suppressed: 0 from 0)

SERVER
valgrind --tool=memcheck ./t3server

valgrind --tool=memcheck ./t3server
==27677== Memcheck, a memory error detector
==27677== Copyright (C) 2002-2013, and GNU GPL'd, by Julian Seward et al.
==27677== Using Valgrind-3.10.0 and LibVEX; rerun with -h for copyright info
==27677== Command: ./t3server
==27677== 
Tic Tac Toe: server online...  waiting for client ... 
...
==27677== 
==27677== HEAP SUMMARY:
==27677==     in use at exit: 0 bytes in 0 blocks
==27677==   total heap usage: 5 allocs, 5 frees, 96 bytes allocated
==27677== 
==27677== All heap blocks were freed -- no leaks are possible
==27677== 
==27677== For counts of detected and suppressed errors, rerun with: -v
==27677== ERROR SUMMARY: 0 errors from 0 contexts (suppressed: 0 from 0)
(16:31) trazzidieg@taj-mahal: /u/students/trazzidieg/Downloads > 

CLIENT
valgrind --tool=memcheck ./t3client

==27679== Memcheck, a memory error detector
==27679== Copyright (C) 2002-2013, and GNU GPL'd, by Julian Seward et al.
==27679== Using Valgrind-3.10.0 and LibVEX; rerun with -h for copyright info
==27679== Command: ./t3client
==27679== 
How big is your board?

...
You won!
==27679== 
==27679== HEAP SUMMARY:
==27679==     in use at exit: 0 bytes in 0 blocks
==27679==   total heap usage: 0 allocs, 0 frees, 0 bytes allocated
==27679== 
==27679== All heap blocks were freed -- no leaks are possible
==27679== 
==27679== For counts of detected and suppressed errors, rerun with: -v
==27679== ERROR SUMMARY: 0 errors from 0 contexts (suppressed: 0 from 0)\end{lstlisting}
\end{document}